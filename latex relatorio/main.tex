\documentclass{article}

% Language setting
% Replace `english' with e.g. `spanish' to change the document language
\usepackage[brazilian]{babel}

% Set page size and margins
% Replace `letterpaper' with `a4paper' for UK/EU standard size
\usepackage[a4paper,top=2cm,bottom=2cm,left=3cm,right=3cm,marginparwidth=1.75cm]{geometry}

% Useful packages (pode incluir mais pacotes se quiser)
\usepackage{amsmath}
\usepackage{graphicx}
\usepackage[colorlinks=true, allcolors=blue]{hyperref}
\usepackage[round]{natbib}
\usepackage[color=red!20]{todonotes}
\usepackage{indentfirst}
\usepackage{float}
\linespread{1.2}

\title{Template para escrita do relatório}
\date{}
\begin{document}
\maketitle

\begin{tabular}{lp{10cm}}
     \textbf{Integrante(s):}& Listar os integrantes\\
     \textbf{E-mail(s):} & e-mail de cada integrante\\
\end{tabular}

\begin{abstract}
Escreva aqui um breve resumo do que foi desenvolvido no projeto, incluindo a contribuição de cada integrante do grupo para sua conclusão. O resumo deve estar inteiramente contido na primeira página e ter no máximo 280 palavras (preferencialmente com menos de 200).\\  
Este \textit{template} deve ser utilizado para a entrega (apesar de ser recomendado, não é obrigatório fazer o relatório no Overleaf, mas ele deve seguir a estrutura apresentada neste \textit{template}).\\  
\textbf{O relatório deve ter no máximo 6 páginas e no mínimo 2, incluindo a página de resumo e as tabelas necessárias para a análise feita pelo grupo.}

\end{abstract}

\newpage

\section{Resultados Obtidos}
\label{sec:intro}

Esta seção deve conter toda a análise feita pelo grupo com base nos testes computacionais realizados.

Todo o código desenvolvido deve estar apenas no arquivo \textit{.ipynb} elaborado pelo grupo. Não insira nenhum código neste relatório, apenas as tabelas necessárias para a análise dos testes executados. Para os dois primeiros testes, os resultados devem ser exibidos também somente no arquivo \textit{.ipynb}. 

\subsection{Teste Computacional 1}

Escreva aqui a análise feita para o primeiro teste (nesta análise não é necessário apresentar nenhuma tabela, apenas um texto explicando o que foi observado).

\subsection{Teste Computacional 2}

Escreva aqui a análise feita para o segundo teste (nesta análise não é necessário apresentar nenhuma tabela, apenas um texto explicando o que foi observado).

\subsection{Teste Computacional 3}

Escreva aqui a análise feita para o terceiro teste (nesta análise apresente uma tabela contendo o valor da função objetivo, o gap e o tempo computacional para cada variação de parâmetros).

Para referenciar a tabela use (Tabela \ref{tab:configuracoes}).

\begin{table}[H]
	\centering
	\caption{Exemplo de Tabela 1 (lembre-se de alterar este título)}
		\begin{tabular}{c c c c} \hline
	 	Configuração & Melhor Solução & \% Gap & Tempo de Execução (s) \\ \hline
		  ($\alpha$ = 2)($\beta$ = 0.05) & 38917.4 & 0.00 & 67.44 \\
		($\alpha$ = 2)($\beta$ = 0) & 38898.2 & 0.00 & 88.05 \\ 
		($\alpha$ = 1.5)($\beta$ = 0.05) & 37958.7 & 0.00 & 52.53 \\
		($\alpha$ = 1.5)($\beta$ = 0) & 37957.7 & 0.00 & 33.86 \\  \hline
		\end{tabular}
  \label{tab:configuracoes}
\end{table}

\subsection{Teste Computacional 4}

Escreva aqui a análise feita para o quarto teste (nesta análise apresente uma tabela contendo o valor da função objetivo, o gap e o tempo computacional para cada instância).

Para referenciar a tabela use (Tabela \ref{tab:teste4}).

\begin{table}[H]
	\centering
	\caption{Exemplo de Tabela 2 (lembre-se de alterar este título)}
		\begin{tabular}{c c c c} \hline
	 	Instância & Melhor Solução & \% Gap & Tempo de Execução (s) \\ \hline
		1  & 38804.4 & 0.00 & 59.19 \\
		2 & 38473.0 & 0.00 & 41.74 \\ 
		3 & 37958.7 & 0.00 & 52.30 \\
		4 & 40386.7 & 0.00 & 4.15 \\  \hline
		\end{tabular}
  \label{tab:teste4}
\end{table}

\subsection{Teste Computacional 5}

Escreva aqui a análise feita para o quinto teste (nesta análise apresente uma tabela contendo o valor da função objetivo, o gap e o tempo computacional para cada instância).

\subsection{Teste Computacional 6}

Escreva aqui a análise feita para o sexto teste (nesta análise apresente uma tabela contendo o valor da função objetivo, o gap e o tempo computacional para cada instância).

\section{Conclusão}
\label{sec:conclusao}

A conclusão deve contextualizar os resultados obtidos e o que foi possível concluir com o desenvolvimento do trabalho. 

Não é necessário fazer uma conclusão muito elaborada, apenas escreva o que foi possível perceber com os resultados obtidos na seção anterior (Escreva no máximo 200 palavras).

% \bibliographystyle{plainnat}
% \bibliography{references}

\end{document}
